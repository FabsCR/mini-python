\documentclass{article}
\usepackage{amsmath, amsthm}
\usepackage{graphicx}

\title{Análisis Sintáctico del Compilador Mini-Python}
\author{
    Fabián Fernández \and Kevin Jiménez \and Justin Martínez
}
\date{Compiladores e Intérpretes\\ Profesor: Oscar Mario Víquez Acuña}

\begin{document}

% Portada
\maketitle
\newpage

% Índice
\tableofcontents
\newpage

\section{Introducción}
Un compilador/intérprete es una herramienta clave para cualquier estudiante de computación. En este proyecto se desarrollará un Analizador Sintáctico para el lenguaje Mini-Python, utilizando la herramienta ANTLR4 con énfasis en el lenguaje de programación C\#. El objetivo es comprender mejor los aspectos relacionados con el análisis sintáctico y la creación de un Árbol de Sintaxis Abstracta (AST).

\section{Análisis del Lenguaje}
En esta sección se detalla el análisis de la gramática del lenguaje Mini-Python. Se discutirán las restricciones impuestas por la gramática, las construcciones permitidas, y las limitaciones.

\subsection{Gramática}
La gramática del lenguaje Mini-Python es la siguiente:
\begin{verbatim}
Program := Program MainStatement | MainStatement   
MainStatement := DefStatement | AssignStatement  
Statement := DefStatement | IfStatement | ReturnStatement | ...
\end{verbatim}

\subsection{Tokens}
La siguiente tabla describe los tokens identificados para el análisis léxico:
\begin{center}
    \begin{tabular}{|c|c|}
        \hline
        Token & Descripción \\
        \hline
        IF & Palabra reservada ``if'' \\
        PLUS & Operador suma ``+'' \\
        ... & ... \\
        \hline
    \end{tabular}
\end{center}

\section{Soluciones e Implementación}
En esta sección se detalla el proceso de implementación del proyecto. Esto incluye la adaptación del código Java presentado en clase a C\#, así como la implementación de las funcionalidades del Scanner y el Parser utilizando ANTLR4.

\subsection{Scanner}
El scanner se encarga de identificar los diferentes tokens presentes en el archivo de entrada. Los comentarios y caracteres ignorados, como los saltos de línea y espacios en blanco, son correctamente manejados.

\subsection{Parser}
El parser se ha implementado utilizando un enfoque de descenso recursivo, aprovechando la estructura del lenguaje Mini-Python para identificar errores sintácticos.

\section{Resultados Obtenidos}
A continuación, se presentan los resultados obtenidos al compilar diversos programas de prueba en Mini-Python. Se resaltan las áreas que funcionan correctamente y los problemas aún pendientes por resolver.

\begin{center}
    \begin{tabular}{|c|c|}
        \hline
        Funcionalidad & Estado \\
        \hline
        Análisis léxico & Completado \\
        Análisis sintáctico & Completado \\
        AST & Parcialmente completado \\
        ... & ... \\
        \hline
    \end{tabular}
\end{center}

\section{Conclusiones}
Este proyecto ha permitido una mayor comprensión del proceso de análisis sintáctico y la creación de un AST en un compilador. A través del uso de ANTLR4 y C\#, se lograron implementar las funcionalidades clave del compilador Mini-Python.

\section{Bibliografía}
\begin{thebibliography}{9}
    \bibitem{antlr} Terrence Parr. \textit{The Definitive ANTLR 4 Reference}. 2013.
    \bibitem{python} Python Documentation, \url{https://docs.python.org/3/}.
\end{thebibliography}

\end{document}
